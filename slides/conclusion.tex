\section{Conclusion}

% \begin{frame}{Outline}
% \tableofcontents[currentsection]
% \end{frame}

\begin{frame}{Conclusion}

\begin{itemize}
\item Motivated by the understanding that flat minima help generalization, we propose adversarial model perturbation (AMP) as an efficient regularization scheme.
\item We theoretically justify that AMP is capable of finding flatter local minima, thereby improving generalization.
\item Extensive experiments on the benchmark datasets demonstrate that AMP achieves the best performance among the compared regularization schemes on various modern neural network architectures.
\end{itemize}

ArXiv: \url{https://arxiv.org/abs/2010.04925}

Code: \url{https://github.com/hiyouga/AMP-Regularizer}

\end{frame}
